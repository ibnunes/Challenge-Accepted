\chapter{Implementação}
% OU \chapter{Trabalhos Relacionados}
% OU \chapter{Engenharia de Software}
% OU \chapter{Tecnologias e Ferramentas Utilizadas}
\label{ch::implementacao}

\section{Introdução}
\label{sec::implementacao:intro}

A fase de implementação envolveu a execução paralela de diferentes tarefas pelos vários elementos da equipa. Este Capítulo aborda em particular os seguintes aspetos desta fase do projeto:

\begin{itemize}
    \item Interface (Secção \ref{sec::implementacao:tui}): aborda as escolhas feitas na construção da \acf{TUI};
    \item Escolhas de implementação (Secção \ref{sec::implementacao:escolhas}): explica as decisões feitas durante a implementação do código-fonte;
    \item Detalhes de implementação (Secção \ref{sec::implementacao:detalhes}): explora os detalhes mais importantes do código-fonte.
\end{itemize}

Adicionalmente, são descritos o malual de instalação (Secção \ref{sec::implementacao:maninstall}) e o manual de utilização (Secção \ref{sec::implementacao:manuser}).


\section{Interface --- \ac{TUI}}
\label{sec::implementacao:tui}

Para a interface foram analisadas primeiramente duas opções existentes em bibliotecas do \textit{Python}: \textit{npyscreen} e \textit{picotui}. Contudo, ambas revelaram ter uma curva de aprendizagem que não compensaria face às restantes tarefas a realizar na elaboração do projeto.

Neste sentido, a \ac{TUI} segue uma filosofia minimalista e clássica de leitura de dados introduzidos por parte do utilizador, incluindo as opções dos menus para navegação.

Não obstante, a aplicação segue um código de cores para diferenciar os diferentes tipos de informação dados ao utilizador (Tabela \ref{tab::cores}).

\begin{table}[!htbp]
    \centering
    \begin{tabular}{>{\itshape}l l p{1cm}}
        \toprule
        \normalfont{\bfseries Cor} & \normalfont{\bfseries Utilização} & \\
        \midrule
        Vermelho & Mensagem de erro (\textit{error})       & \cellcolor[rgb]{1., 0., 0.} \\
        Amarelo  & Mensagem de aviso (\textit{warning})    & \cellcolor[rgb]{0.941, 0.886, 0.23} \\
        Verde    & Mensagem de sucesso (\textit{success})  & \cellcolor[rgb]{0., 1., 0.} \\
        Ciano    & Informação da aplicação (\textit{info}) & \cellcolor[rgb]{0.239, 0.843, 0.941} \\
        Cinza    & Mensagens de \textit{debug} (reservado) & \cellcolor[rgb]{0.4, 0.4, 0.4} \\
        \bottomrule
    \end{tabular}
    \caption[Cores por tipo de mensagem]{Palete de cores utilizada por cada tipo de mensagem dada ao utilizador.}
    \label{tab::cores}
\end{table}


\section{Escolhas de Implementação}
\label{sec::implementacao:escolhas}

De entre as escolhas efetuadas durante a implementação da aplicação, três em
particular destacam-se:

\begin{itemize}
    \item \textbf{\textit{Python}:}\\
        A seleção da linguagem de programação teve como critérios ser de alto nível, fornecer bibliotecas atualizadas de segurança e criptografia, e disponibilizar \textit{frameworks} acessíveis para a criação de um \textit{webservice}. A escolha final recaiu, portanto, na linguagem \textit{Python}.
    
    \item \textbf{\textit{MariaDB}:}\\
        Uma vez que o grupo se encontra familiarizado com bases de dados relacionais, e sendo este um modelo adequado para os dados a guardar, optou-se por uma solução \textit{open-source}, estável, atualizada e com reputação no mundo profissional: \textit{MariaDB}.
    
    \item \textbf{\textit{Flask}:}\\
        Após a escolha da linguagem \textit{Python}, a \textit{framework} que rapidamente se destacou para a criação do \textit{webservice} foi o \textit{Flask}. Destaca-se o facto da curva de aprendizagem desta ferramenta ser curta.
\end{itemize}


\section{Detalhes de Implementação}
\label{sec::implementacao:detalhes}

% TODO: Detalhes

%Porquanto a implementação no seu todo desse facilmente origem a um vasto documento técnico, três situações destacaram-se:

\subsection{Navegação}
\label{ssec::implementacao:detalhes:nav}

A navegação é feita com menus, os quais são objetos instanciados da classe \verb|Menu|, criada para este projeto. Esta classe permite fazer a gestão automatizada de menus, sendo uma função invocada quando uma dada opção válida do menu é selecionada.

Tal permite tirar partido da \textit{stack} de invocações do sistema operativo, levando a que a navegação entre menus seja resultado da hierarquia destas chamadas.


\subsection{Estruturação dos \textit{packages} e classes}
\label{ssec::implementacao:detalhes:estrutura}

A linguagem \textit{Python}, porquanto não siga a mesma filosofia de \textit{packages} da linguagem \textit{Java}, tem mecanismos que o permitem simular até certo ponto.

Desta forma, os vários ficheiros que constituem bibliotecas da aplicação foram distribuídas por pastas, ficando o interpretador \textit{Python} a conhecer a sua localização graças à criação de ficheiros \verb|__init__.py| na raiz de cada pasta que se queira considerar como uma \textit{package}:

\begin{itemize}
    \item \verb|challenge|: Classes com métodos estáticos que implementam elementos da \ac{TUI} específicos ao programa;
    \item \verb|dbhelper|: Comunicação com o \textit{webservice};
    \item \verb|login|: Gestão de \textit{login}, registo e sessão local de um utilizador;
    \item \verb|tui|: Elementos essenciais à \ac{TUI};
    \item \verb|utils|: Utilitários variados transversais às restantes bibliotecas.
\end{itemize}

Por outro lado, diferentes objetivos e funcionalidades são divididos em classes distintas com métodos que permitem melhor abstrair, sequencialmente, as operações que se tornam progressivamente de mais ``baixo nível'' (i.e., da \ac{TUI} à comunicação com o \textit{webservice}).


\subsection{Cifras e algoritmos de \textit{hash}}
\label{ssec::implementacao:detalhes:cifras}

Os algoritmos implementados foram reunidos nas classes \verb|Cypher| e \verb|Hash| do \textit{package} \verb|client.util|. Os respetivos métodos foram implementados com recurso a bibliotecas disponibilizadas pela ferramenta \textit{pip3}, as quais têm as vantagens de ser atuais, \textit{open-source} e vastamente utilizadas pela comunidade criptográfica e de \textit{development}.


\subsection{\textit{Webservice}}
\label{ssec::implementacao:detalhes:webservice}

A fim de o cliente comunicar com o serviço alojado na \textit{cloud} de forma encriptada e segura, foi criado um \textit{webservice} com recurso à \textit{framework} \textit{\bfseries Flask}. Este escuta por pedidos no porto $443$ (protocolo \ac{HTTPS}) e, conforme o método do pedido (\verb|GET|, \verb|POST| ou \verb|PATCH|) e o \ac{URL} por onde este é efetuado, o \textit{webservice} executa uma ação programada e devolve uma resposta com o resultado do pedido.

Os dados são bidirecionalmente encapsulados no formato \ac{JSON}. O \textit{webservice} em particular devolve uma resposta com indicação de sucesso no pedido (fornecendo os dados respetivos) ou de erro (com a respetiva mensagem associada).

O \textit{webservice}, a cada pedido (e após confirmar que se trata de um pedido efetuado por uma \textit{app} autorizada (Secção \ref{ssec::implementacao:detalhes:auth})), abre uma conexão com a \acl{BD} e processa os resultados da \textit{query} de forma a devolvê-los num formato aceite pelo cliente (\textit{string} ou dicionário).

O \textit{webservice} encontra-se atualmente alojado no domínio \url{https://chapted.igornunes.com}.


\subsection{Serviço de autenticação de clientes}
\label{ssec::implementacao:detalhes:auth}

Um cliente apenas pode realizar pedidos ao \textit{webservice} caso seja uma \textit{app} autorizada. Para tal, o cliente deve enfrentar um desafio proposto pelo \textit{webservice} a fim de a autenticar.

Para tal, é gerado um \textit{header} no pedido \ac{HTTPS} com os dados necessários para o \textit{webservice} constatar que o cliente superou o desafio e é, portanto, válido.

O algoritmo envolve um ID da aplicação cliente, uma chave associada, e a geração do \textit{hash} do corpo do pedido \ac{HTTPS}, assim como um \textit{timestamp} e um \textit{nonce}. No final do desafio, a \textit{signature} deve coincidir com o \textit{HMAC} dos dados anteriores processados.

A não presença deste \textit{header} (Figura \ref{fig::auth-fail}) ou a presença de dados que não permitem resolver o desafio do \textit{webservice} invalidam o acesso a este, sendo o pedido recusado.

O algoritmo de \textit{hash} utilizado no processo é o \textit{SHA256}.

\begin{figure}[!htbp]
    \centering
    \includegraphics[scale=0.75]{authfail}
    \caption[Falha de autenticação]{Exemplo de falha de autenticação ao aceder através de um \textit{browser}.}
    \label{fig::auth-fail}
\end{figure}



\section{Manual de Instalação}
\label{sec::implementacao:maninstall}

Para a primeira utilização da aplicação é necessário, no terminal, mudar para o diretório onde se encontra o ficheiro \verb|setup.py| e executar o comando \verb|pip3 install .| a fim de instalar as dependências necessárias do \textit{Python}.

Após este passo, o programa pode ser utilizado conforme indicado no manual de utilização (Capítulo \ref{sec::implementacao:manuser}).


\section{Manual de Utilização}
\label{sec::implementacao:manuser}

Ao correr a aplicação \appname~(com o comando \verb|python3 app.py|), é apresentado ao utilizador um primeiro menu, no qual pode escolher entre criar uma conta (caso seja a primeira utilização da plataforma), fazer \textit{login}, ver informações da aplicação ou sair do programa.  Para o caso de criar conta, é-lhe pedido o fornecimento de um \textit{email} válido e uma \textit{password}, a qual é confirmada uma segunda vez.

Após o \textit{login}, é apresentada ao utilizador uma \textit{homepage} onde o utilizador tem acesso às várias funcionalidades da aplicação: propor desafio, responder a desafio, listar desafios e ver \textit{scoreboard} dos jogadores.

Um utilizador poderá submeter vários tipos de desafios, mas não poderá responder aos que por ele foram criados. Conforme o utilizador responda com sucesso aos desafios propostos, é-lhe atribuída uma pontuação e o mesmo poderá verificar em que lugar se encontra no \textit{scoreboard}.

Está disponível com o argumento \verb|--debug| (ou \verb|-d|) um modo de \textit{debugging} para desenvolvimento.


\section{Conclusões}
\label{sec::implementacao:concs}

A exposição dos pontos mais importantes relacionados com a fase de implementação do serviço \appname~permitiu à equipa fazer uma retrospetiva do seu trabalho e perceber quais foram os pontos fortes e os pontos fracos do resultado final. Tal abre a porta para a fase de reflexão crítica.

