\chapter{Introdução}
\label{chap:intro}

\section{Descrição da proposta}
\label{sec::intro:descricao}

%TODO: Melhorar introdução

A plataforma \appname~tem como \textbf{objetivo} permitir aos seus utilizadores publicar e resolver vários desafios de criptografia.

% O projeto foi proposto pelo Professor Doutor Pedro Ricardo Morais Inácio no âmbito da cadeira Segurança Informática, lecionada pelo mesmo.


\section{Constituição do grupo}
\label{sec::intro:grupo}

O presente projeto foi realizado pela equipa \textit{C-Team}, constituída pelos elementos listados na Tabela \ref{tab::c-team}.

\begin{table}[!h]
	\centering
	\begin{tabular}{c l >{\itshape}l}
		\toprule
		\textbf{N\textordmasculine} & \textbf{Nome} & \normalfont\textbf{Alcunha} \\
		\midrule
		38950 & Diogo José Real Lavareda    & Lavareda \\
		39392 & Joana Elias Almeida         & Joaninha \\
		41266 & Diogo Castanheira Simões    & Ash      \\
		41358 & Beatriz Tavares da Costa    & Bea      \\
		41381 & Igor Cordeiro Bordalo Nunes & Etileno  \\
		\bottomrule
	\end{tabular}
	\caption[Constituição da equipa \textit{C-Team}]{Constituição da equipa \textit{C-Team}.}
	\label{tab::c-team}
\end{table}



\section{Organização do Documento}
\label{sec::intro:organizacao}

De modo a refletir o projeto realizado, este relatório encontra-se estruturado em cinco capítulos, nomeadamente:

\begin{enumerate}
\item No primeiro capítulo --- \textbf{Introdução} --- são apresentados o projeto, os seus objetivos, a equipa desenvolvedora e a respetiva organização do relatório.

\item No segundo capítulo --- \textbf{Engenharia de Software} --- são elaborados os diagramas de casos de uso da aplicação que orientam a respetiva implementação.

\item No terceiro capítulo --- \textbf{Implementação} --- são descritas as escolhas e os detalhes de implementação da aplicação, bem como as tecnologias utilizadas durante o seu desenvolvimento.

\item No quarto capítulo --- \textbf{Reflexão Crítica e Problemas Encontrados} --- são indicados os objetivos alcançados, quais as tarefas realizadas por cada membro do grupo, assim como são expostos os problemas enfrentados e é feita uma reflexão crítica sobre o trabalho.

\item No quinto capítulo --- \textbf{Conclusões e Trabalho Futuro} --- são analisados os conhecimentos adquiridos ao longo do desenvolvimento do projeto e, em contrapartida, o que não se conseguiu alcançar e que poderá ser explorado futuramente.
\end{enumerate}
