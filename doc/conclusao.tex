\chapter{Conclusões e Trabalho Futuro}
\label{chap:conc-trab-futuro}

\section{Conclusões Principais}
\label{sec:conc-princ}

Este projeto e a sua elaboração permitiu adquirir uma série de novos conhecimentos e técnicas na área de criptografia, segurança e respetiva aplicação em sistemas distribuídos, assim como abriu horizontes para novos métodos e técnicas de desenvolvimento.

Apesar de certos aspetos que nos pareciam significativos para o resultado final da aplicação não terem sido implementados neste projeto (nomeadamente a assinatura digital com \ac{RSA}), considera-se que o trabalho necessário e essencial para o mesmo foi concluído com sucesso, acrescentando ainda algumas cifras adicionais (nomeadamente Cifra de César, Vigenere e \textit{One Time Pad}).

A investigação necessária para a realização deste projeto servirá como uma mais-valia para o desenvolvimento de outras aplicações cuja segurança seja uma prioridade imprescindível. Para além disso, levou a um maior entendimento da matéria teórica que é abrangida no âmbito da Unidade Curricular (UC) do projeto, através da sua aplicação de forma prática em todas as etapas do processo de desenvolvimento e implementação.

% É de notar que o planeamento e gestão do mesmo projeto permitiu não só aprender a utilizar novas ferramentas de planeamento ágil e de cooperação de equipas (\textit{git}), como também facilitar o processo de comunicação e divisão de tarefas perante o contexto socioeconómico em que nos encontramos.


\section{Trabalho Futuro}
\label{sec:trab-futuro}

Uma melhor integração do \textit{webservice} com a \acl{BD} é relevante a fim de evitar acessos à última fora do contexto do \textit{webservice}. A par desta potencial falha de segurança, a implementação da assinatura com \ac{RSA} é de primeira ordem.

De igual forma, a adição de novos algoritmos criptográficos tornaria o serviço mais apelativo, nomeadamente com uma interface gráfica ou uma \textit{web app} com \textit{layout} mais apelativo do que um programa \acf{TUI}.

