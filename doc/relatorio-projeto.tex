\documentclass[12pt,a4paper]{memoir}
% \documentclass[titlepage,12pt,a4paper]{book}

% substituir linha seguinte por 
% \usepackage[english]{babel} 
% se o relatório for escrito na língua inglesa.
\usepackage[portuguese]{babel}

% \usepackage[utf8]{inputenc}
\usepackage[T1]{fontenc}

\usepackage{makeidx}
\usepackage{xspace}
\usepackage{graphicx,color,times}
\usepackage{fancyhdr}
% \usepackage{pxfonts}
% \usepackage{times}
% \usepackage{mathptm}
% \usepackage{amssymb}
% \usepackage{amsfonts}

\usepackage{amsmath}
\usepackage{latexsym}
\usepackage[printonlyused]{acronym}
\usepackage{float}
\usepackage{listings}
\usepackage{tocbibind}
\usepackage{natbib}
\usepackage{hyperref}

% \usepackage{glossaries}
% \makeglossaries

% \renewcommand{\ttdefault}{phv}

\pagestyle{fancy}
\renewcommand{\chaptermark}[1]{\markboth{#1}{}}
\renewcommand{\sectionmark}[1]{\markright{\thesection\ #1}}
\fancyhf{} \fancyhead[LE,RO]{\bfseries\thepage}
\fancyhead[LO]{\bfseries\rightmark}
\fancyhead[RE]{\bfseries\leftmark}
\renewcommand{\headrulewidth}{0.5pt}
\renewcommand{\footrulewidth}{0pt}
\setlength{\headheight}{15.6pt}
\setlength{\marginparsep}{0cm}
\setlength{\marginparwidth}{0cm}
\setlength{\marginparpush}{0cm}
\addtolength{\hoffset}{-1.0cm}
\addtolength{\oddsidemargin}{\evensidemargin}
\addtolength{\oddsidemargin}{0.5cm}
\addtolength{\evensidemargin}{-0.5cm}


\usepackage{fix-cm}
\usepackage{fourier}
\usepackage[scaled=.92]{helvet}
\definecolor{ChapGrey}{rgb}{0.6,0.6,0.6}
\newcommand{\LargeFont}{
  \usefont{\encodingdefault}{\rmdefault}{b}{n}
  \fontsize{60}{80}\selectfont\color{ChapGrey}
  }
\makeatletter
\makechapterstyle{GreyNum}{
  \renewcommand{\chapnamefont}{\large\sffamily\bfseries\itshape}
  \renewcommand{\chapnumfont}{\LargeFont}
  \renewcommand{\chaptitlefont}{\Huge\sffamily\bfseries\itshape}
  \setlength{\beforechapskip}{0pt}
  \setlength{\midchapskip}{40pt}
  \setlength{\afterchapskip}{60pt}
  \renewcommand\chapterheadstart{\vspace*{\beforechapskip}}
  \renewcommand\printchaptername{
  \begin{tabular}{@{}c@{}}
    \chapnamefont \@chapapp\\}
    \renewcommand\chapternamenum{\noalign{\vskip 2ex}}
    \renewcommand\printchapternum{\chapnumfont\thechapter\par}
    \renewcommand\afterchapternum{
  \end{tabular}
  \par\nobreak\vskip\midchapskip}
  \renewcommand\printchapternonum{}
  \renewcommand\printchaptertitle[1]{
  {\chaptitlefont{##1}\par}}
  \renewcommand\afterchaptertitle{\par\nobreak\vskip \afterchapskip}
}
\makeatother
\chapterstyle{GreyNum}

\setcounter{tocdepth}{3}
\setsecnumdepth{subsubsection}

\renewcommand{\ttdefault}{lmtt}


% NEW COLORS
\definecolor{dark}{gray}{0.25}
\definecolor{lgray}{gray}{0.9}
\definecolor{dkblue}{rgb}{0,0.13,0.4}
\definecolor{dkgreen}{rgb}{0,0.6,0}
\definecolor{gray}{rgb}{0.5,0.5,0.5}
\definecolor{mauve}{rgb}{0.58,0,0.82}

\lstset{ %
  language=C,                    basicstyle=\footnotesize,
  numbers=none,                  numberstyle=\tiny\color{gray}, 
  stepnumber=1,                  numbersep=5pt,
  backgroundcolor=\color{white}, showspaces=false,
  showstringspaces=false,        showtabs=false,
  frame=single,                  rulecolor=\color{black},
  tabsize=2,                     captionpos=b,
  breaklines=true,               breakatwhitespace=false,
  title=\lstname,                keywordstyle=\color{blue},
  commentstyle=\color{dkgreen},  stringstyle=\color{mauve},
  escapeinside={\%*}{*)},        morekeywords={*},
  belowskip=0cm
}

\renewcommand{\lstlistingname}{Excerto de Código}
\renewcommand{\lstlistlistingname}{Lista de Excertos de Código}

\renewcommand{\today}{\day \ifcase \month \or Janeiro\or Fevereiro\or Março\or %
Abril\or Maio\or Junho\or Julho\or Agosto\or Setembro\or Outubro\or Novembro\or %
Dezembro\fi de \number \year} 



\begin{document}


\thispagestyle{empty}
\setcounter{page}{-1}

\begin{center}
\begin{Huge}
\textbf{Universidade da Beira Interior}
\end{Huge}
\end{center}

\begin{center}
\begin{Huge}
Departamento de Informática
\end{Huge}
\end{center}

\vspace{0,07cm}
\begin{figure}[!htb]
\centering
\includegraphics[width=191pt]{ubi-fe-di.png}
\end{figure}

\vspace{0.5cm}
\begin{center}
\begin{Large}
\textbf{ C-Team: \emph{CHALLENGE-ACCEPTED}}
\end{Large}
\end{center}


\vspace{0.3cm}
\begin{center}
\begin{normalsize}
\begin{large}
Elaborado por:
\end{large}
\end{normalsize}
\end{center}

\vspace{0.1cm}
\begin{center}
\begin{large}
\textbf{38950 - Diogo José Real Lavareda}
\end{large}
\end{center}
\begin{center}
\begin{large}
\textbf{39392 - Joana Elias Almeida}
\end{large}
\end{center}
\begin{center}
\begin{large}
\textbf{41266 - Diogo Castanheira Simões}
\end{large}
\end{center}
\begin{center}
\begin{large}
\textbf{41358 - Beatriz Tavares da Costa}
\end{large}
\end{center}
\begin{center}
\begin{large}
\textbf{41381 - Igor Cordeiro Bordalo Nunes}
\end{large}
\end{center}

\vspace{0,3cm}
\begin{center}
\begin{normalsize}
\begin{large}
Orientador:
\end{large}
\end{normalsize}
\end{center}

\vspace{0.1cm}
\begin{center}
\begin{large}
\textbf{Professor Doutor Pedro Ricardo Morais Inácio}
\end{large}
\end{center}



\vspace{0.5cm}
\begin{center}
\begin{normalsize}
\today
\end{normalsize}
\end{center}


\clearpage{\thispagestyle{empty}\cleardoublepage}

\frontmatter

\chapter*{Resumo}
\label{chap:ack}



\clearpage{\thispagestyle{empty}\cleardoublepage}


\tableofcontents

\clearpage{\thispagestyle{empty}\cleardoublepage}

\listoffigures

% #   ATENÇÃO
% Se não existirem tabelas, comentar as duas linhas seguintes
\clearpage{\thispagestyle{empty}\cleardoublepage}
\listoftables

% #   ATENÇÃO
% Se existirem trechos de código, descomentar as seguintes linhas
% \clearpage{\thispagestyle{empty}\cleardoublepage}
% \lstlistoflistings

\clearpage{\thispagestyle{empty}\cleardoublepage}
\chapter*{Acrónimos}
\label{chap:acro}

% #   ATENÇÃO
% A lista de acrónimods deve ser ordenada alfanumericamente.
% Estrangeirismos devem ser realçados em itálico.
% Se o relatório for escrito em Inglês, uma palavra portuguesa é um estrangeirismo.

% O maior acrónimo deve ser colocado neste ponto (reparar que CFIUTE é maior que TCP!).
%               vvvvvv
\begin{acronym}[CFIUTE]

  \acro{CFIUTE}{Centro de Formação Interação UBI Tecido Empresarial}
  \acro{RSA}{Rivest-Shamir-Adleman}
  \acro{TCP}{\emph{Transmission Control Protocol}}
  \acro{UBI}{Universidade da Beira Interior}

\end{acronym}

% \clearpage{\pagestyle{empty}\cleardoublepage}
% \include{glossario}

\clearpage{\thispagestyle{empty}\cleardoublepage}

\mainmatter
\acresetall
\chapter{Introdução}
\label{chap:intro}

\section{Descrição da proposta}
\label{sec::intro:descricao}
A plataforma \textit{CHALLENGE-ACCEPTED} tem como \textbf{objetivo} permitir aos seus utilizadores publicar e resolver vários desafios de encriptação.

O projeto foi proposto pelo Professor Doutor Pedro Ricardo Morais Inácio no âmbito da cadeira Segurança Informática, lecionada pelo mesmo.


\section{Constituição do grupo}
\label{sec::intro:grupo}

O presente projeto foi realizado pela equipa \textit{C-Team}, constituída pelos elementos listados na Tabela \ref{tab::c-team}.

\begin{table}[!h]
	\centering
	\begin{tabular}{c l >{\itshape}l}
		\toprule
		\textbf{N\textordmasculine} & \textbf{Nome} & \normalfont\textbf{Alcunha} \\
		\midrule
		38950 & Diogo José Real Lavareda    & Lavareda \\
		39392 & Joana Elias Almeida         & Joaninha \\
		41266 & Diogo Castanheira Simões    & Ash      \\
		41358 & Beatriz Tavares da Costa    & Bea      \\
		41381 & Igor Cordeiro Bordalo Nunes & Etileno  \\
		\bottomrule
	\end{tabular}
	\caption[Constituição da equipa \textit{C-Team}]{Constituição da equipa \textit{C-Team}.}
	\label{tab::c-team}
\end{table}



\section{Organização do Documento}
\label{sec::intro:organizacao}
% !POR EXEMPLO!
De modo a refletir o projeto realizado, este relatório encontra-se estruturado em 5 capítulos, nomeadamente:

\begin{enumerate}
\item No primeiro capítulo --- \textbf{Introdução} --- são apresentados o projeto, os seus objetivos, a equipa desenvolvedora e a respetiva organização do relatório.

\item No segundo capítulo --- \textbf{Engenharia de Software} --- são elaborados os diagramas de casos de uso da aplicação que orientam a respetiva implementação.

\item No terceiro capítulo --- \textbf{Implementação} --- são descritas as escolhas e os detalhes de implementação da aplicação, bem como as tecnologias utilizadas durante o seu desenvolvimento.

\item No quarto capítulo --- \textbf{Reflexão Crítica e Problemas Encontrados} --- são indicados os objetivos alcançados, quais as tarefas realizadas por cada membro do grupo, assim como são expostos os problemas enfrentados e é feita uma reflexão crítica sobre o trabalho.

\item No quinto capítulo --- \textbf{Conclusões e Trabalho Futuro} --- são analisados os conhecimentos adquiridos ao longo do desenvolvimento do projeto e, em contrapartida, o que não se conseguiu alcançar e que poderá ser explorado futuramente.
\end{enumerate}

\clearpage{\thispagestyle{empty}\cleardoublepage}
\include{capitulo-2}
\clearpage{\thispagestyle{empty}\cleardoublepage}
\chapter{Tecnologias e Ferramentas Utilizadas}
% OU \chapter{Trabalhos Relacionados}
% OU \chapter{Engenharia de Software}
% OU \chapter{Tecnologias e Ferramentas Utilizadas}
\label{chap:tecno-ferra}

\section{Introdução}
\label{chap3:sec:intro}
Cada capítulo \underline{intermédio} deve começar com uma breve introdução onde é explicado com um pouco mais de detalhe qual é o tema deste capítulo, e como é que se encontra organizado (i.e., o que é que cada secção seguinte discute).

\section{Secções Intermédias}
\label{chap3:sec:...}

A tabela~\ref{tab:exemplo} serve apenas o propósito da exemplificação de como se fazem tabelas em \LaTeX.
%
\begin{table}
\centering
\begin{tabular}{|c|lr|}
\hline
\textbf{campo 1} & \textbf{campo 2} & \textbf{campo 3}\\
\hline
\hline
14 & 15 & 16 \\
\hline	
13 & 13 & 13 \\
\hline
\end{tabular}
\caption{Esta é uma tabela de exemplo.}
\label{tab:exemplo}
\end{table}

\section{Conclusões}
\label{chap3:sec:concs}
Cada capítulo \underline{intermédio} deve referir o que demais importante se conclui desta parte do trabalho, de modo a fornecer a motivação para o capítulo ou passos seguintes.
\clearpage{\thispagestyle{empty}\cleardoublepage}
\chapter{Reflexão Crítica e Problemas Encontrados}
\label{chap:reflexao}

\section{Introdução}
\label{chap4:sec:intro}

Não obstante o bom planeamento feito a priori na fase de engenharia de \emph{software} (descrito no Capítulo \ref{ch::engsoft}), o projeto \emph{Challenge-Accepted}, tal como qualquer outro na área das \ac{TI}, enfrentou alguns contratempos e, com o tempo ao dispor, não se revelou possível almejar todas as ambições inicialmente imaginadas. É preciso, pois, refletir sobre o desenvolvimento deste projeto. 
    
    Neste Capítulo são, portanto, explorados os seguintes tópicos:
\begin{itemize}
\item Objetivos propostos vs. alcançados (Secção \ref{chap4:sec:opvsoa}): compara os objetivos
inicialmente propostos com aqueles que foram concluídos no projeto final;
\item Divisão de trabalho pelos elementos do grupo (Secção \ref{chap4:sec:divisao}): lista as tarefas
realizadas por cada elemento da equipa;
\item Problemas encontrados (Secção \ref{chap4:sec:problemas}): na sequência da Secção \ref{chap4:sec:opvsoa}, explora
os problemas encontrados durante a implementação da aplicação;
\item Reflexão crítica (Secção \ref{chap4:sec:reflexao}): é feita uma \ac{SWOT} em retrospetiva pela equipa acerca do projeto.
\end{itemize}

\section{Objetivos Propostos vs. Alcançados}
\label{chap4:sec:opvsoa}

\begin{table}[!htbp]
	\centering
	\begin{tabular}{p{.65\textwidth} >{\centering\let\newline\\\arraybackslash\hspace{0pt}}m{.25\textwidth}}
		\toprule
		{\bfseries Objetivo proposto} & {\bfseries Alcançado?} \\
		\midrule
		Registo de utilizadores com representação segura da palavra-passe na base de dados & $\bullet$ \\
		Submissão de desafios do tipo cifra de mensagens, codificando-a em \textit{BASE64} & $\bullet$ \\	
		Cálculo de código de autenticação de mensagens que permite verificar se uma mensagem foi bem decifrada ou não & $\bullet$ \\
		Submissão de desafios do tipo valor de \textit{hash}, codificando-a em \textit{BASE64} & $\bullet$ \\
		Resposta a desafios e verificação do sucesso da tentativa  & $\bullet$ \\
		Desafios com cifras AES-128-ECB, AES-128-CBC e AES-128-CTR & $\bullet$ \\
		Desafios com funções de hash MD5, SHA256 e SHA512          & $\bullet$ \\
		Limite de uma tentativa a cada 15 segundos para os desafios de cifra &  \\
		Verificar se uma mensagem foi bem decifrada através de assinaturas digitais RSA & - \\
		Suporte ao algoritmo El Gamal                                           & \\
		Outro tipo de desafios criptográficos                                   & \\
		\bottomrule
	\end{tabular}
	\caption[Objetivos propostos vs. alcançados]{
		Objetivos propostos e respetiva indicação de sucesso.\\
		\textit{Legenda.} $\bullet$ Alcançado em pleno; $\circ$ Alcançado parcialmente. -- Não alcançado.
	}
	\label{tab::objetivos}
\end{table}



\section{Divisão do Trabalho pelos Elementos do Grupo}
\label{chap4:sec:divisao}

\begin{table}[!htbp]
	\centering
	\begin{tabular}{l c c c c c}
		\toprule
		\textbf{Tarefa}                             & \textbf{DL} & \textbf{JA} & \textbf{DS} & \textbf{BC} & \textbf{IN} \\
		\midrule
		Gestão do projeto                           &             &             &             &              &            \\
		Engenharia de Software (requisitos)         &             &             &             &              &             \\
		Engenharia de Software (diagramas)          &             &             &             &              &             \\
		Instalação infraestrutura de suporte        &             &             &             &              &            \\
	    Desenvolvimento da base de dados            &             &             &             &              &            \\
		\textit{Framework} da aplicação (cliente-servidor)   &             &             &             &              &            \\
		Implementação dos algoritmos de cifra       &             &             &             &              &            \\
		Documentação do código                      &             &             &             &              &             \\
		Gestão do repositório \textit{git}          &             &             &             &              &              \\
		Relatório                                   &             &             &             &              &             \\
		Apresentação                                &             &             &             &              &              \\
		\bottomrule
	\end{tabular}
	\caption[Distribuição de tarefas]{
		Distribuição de tarefas pelos elementos do grupo.\\
		\textit{Legenda.}~%
		$\bullet$ principal responsável; $\circ$ auxiliou.
		DL: Diogo Lavareda; JA: Joana Almeida; DS: Diogo Simões; BC: Beatriz Costa; IN: Igor Nunes.
	}
	\label{tab::divisao-trabalho}
\end{table}

\section{Problemas Encontrados}
\label{chap4:sec:problemas}

\subsection{Rede \ac{eduroam} e certificado \ac{SSL}}
\label{chap4:subsec:eduroam}

\textbf{(Alterar a escrita apenas as ideias a enunciar)}\\
Ao testarmos a aplicação na rede \ac{eduroam} onde o presente trabalho será defendido, deparamo-nos com o problema desta rede bloquear as comunicações para porto 3300 utilizado no servidor.
Tambem devido ao servidor não ter um \ac{FQDN} para aplicar certificado \ac{SSL}, desta forma para minimizar as alterações a estrutura do servidor optou-se por separar o \emph{WebService} da base de dados para que a aplicação possa ser executada no ambiente de rede da \ac{UBI}.
A solução inicial foi alterada, tendo sido separado o \textit{Web Service} da base de dados para efeito da apresentação do trabalho \ref{fig::diagrama-sistema-new}.

\begin{figure}[!htbp]
	\centering
	\includegraphics[scale=0.325]{Imagens/DiagramaActual.png}	\caption[Diagrama da arquitetura do sistema atulizado]{Diagrama da arquitetura do sistema atulizado}
	\label{fig::diagrama-sistema-new}
\end{figure}

\section{Reflexão Crítica}
\label{chap4:sec:reflexao}

\subsection{Pontos Fortes}
\label{chap4:subsec:pontosfortes}

\subsection{Pontos Fracos}
\label{chap4:subsec:pontosfracos}

\subsection{Ameaças}
\label{chap4:subsec:ameacas}

\subsection{Oportunidades}
\label{chap4:subsec:oportunidades}

\section{Conclusões}
\label{chap4:sec:concs}
Cada capítulo \underline{intermédio} deve referir o que demais importante se conclui desta parte do trabalho, de modo a fornecer a motivação para o capítulo ou passos seguintes.
\clearpage{\thispagestyle{empty}\cleardoublepage}
\include{conc-trab-futuro}
\clearpage{\thispagestyle{empty}\cleardoublepage}

% SE EXISTIREM APENDICES, DESCOMENTAR O QUE ESTÁ EM BAIXO
% \appendix
% \include{apendice1}
% \clearpage{\pagestyle{empty}\cleardoublepage}
% \include{continuacao}
% \clearpage{\pagestyle{empty}\cleardoublepage}
% \include{apendice2}
% \clearpage{\pagestyle{empty}\cleardoublepage}
% \include{apendice3}
% \clearpage{\pagestyle{empty}\cleardoublepage}

\backmatter

\bibliographystyle{unsrt}
\bibliography{bibliografia}

\end{document}