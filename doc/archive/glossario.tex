\chapter*{Glossário}
\makeglossaries

\newglossaryentry{.NET Framework}
{
  name={.NET Framework},
  description={É uma plataforma para desenvolvimento e funcionamento de aplicações desenvolvida pela Microsoft.}
}

\newglossaryentry{ASP.NET}
{
  name={ASP .Net},
  description={É uma plataforma da Microsoft para o desenvolvimento de aplicações Web e é o sucessor da tecnologia ASP.}
}

\newglossaryentry{CS}
{
  name={C\#},
  description={Lê-se \textit{C Sharp} e é uma linguagem de programação orientada a objectos, desenvolvida pela Microsoft, inicialmente para a plataforma .NET. O C\# é inspirado na junção entre as linguagens C++ e Java.}
}


\newglossaryentry{Java}
{
  name={JAVA},
  description={É uma linguagem de programação orientada a objectos, desenvolvida pela Sun Microsystems na década de 90. Hoje pertence à empresa Oracle.}
}


\newglossaryentry{OpenDMTP}
{
  name={OpenDMTP},
  description={\textit{Open Device Monitoring and Tracking Protocol} é um protocolo e uma \textit{framework} abertos que permite a comunicação bidireccional entre servidores e clientes através da internet.}
}


\newglossaryentry{OpenGTS}
{
  name={Open GTS},
  description={É o primeiro projecto \textit{Open Source} \textit{Web-Based} para controlo de frotas por GPS.}
}


\newglossaryentry{VS2010}
{
  name={Visual Studio 2010},
  description={\textit{Microsoft Visual Studio 2010} é um sistema de desenvolvimento desenvolvido pela Microsoft e é dedicado ao Framework .NET, que contem um conjunto de ferramentas de desenvolvimento projectadas para auxiliar os programadores a enfrentarem desafios complexos.}
}


\newglossaryentry{WebS}
{
	name={Web Service},
	description={Web services são aplicações modulares auto-descritas e auto-contidas, que permitem a integração de sistemas e a comunicação entre aplicações de diferentes tipos.}
}


\newglossaryentry{WebBased}
{
	name={Web Based},
	description={Aplicação desenvolvida para a Web.}
}

\newglossaryentry{Roaming}
{
	name={Roaming},
	description={Define a possibilidade de um utilizador de uma determinada rede obter rede/conecção fora da área geográfica onde foi registado.}
}


\newglossaryentry{Smartphone}
{
	name={Smartphone},
	description={Smartphone é um telefone móvel que contem muitas das principais tecnologias de comunicação e serviços que existem nos computadores pessoais, como acesso a e-mails, serviços de mensagens instantâneas, internet, GPS, entre outros.}
}

\newglossaryentry{TCPIP}
{
	name={TCP/IP},
	description={É um conjunto de protocolos de comunicação entre computadores ligados rede. O nome TCP/IP surge da união entre dois protocolos: o TCP (Transmission Control Protocol) e o protocolo IP (Internet Protocol).}
}

\newglossaryentry{Firewall}
{
	name={Firewall},
	description={É o nome criado para definir um dispositivo para uma rede de computadores que tem como objectivo criar uma política de segurança num determinado ponto de controlo da rede.}
}

\newglossaryentry{JavaScript}
{
	name={JavaScript},
	description={É uma linguagem de programação baseada na linguagem de programação ECMAScript. Actualmente é a linguagem de programação mais utilizada em \textit{``Client-Side''} nos \textit{browsers}.}
}

\newglossaryentry{Flash}
{
	name={Flash},
	description={Desenvolvido pela Macromedia, o Flash é um software utilizado para criação de animações interactivas que funcionam incorporadas em \textit{Browsers}, \textit{Desktop}, \textit{Smartphones}, \textit{Tablets}, e Televisores.}
}


\newglossaryentry{StoredProcedure}
{
	name={Stored Procedure },
	description={É o nome dado a um conjunto de comandos numa base de dados de forma a simplificar a sua utilização.}
}

\newglossaryentry{SQLS}
{
	name={SQL Server 2008},
	description={É um sistema de gestão de base de dados relacional criado pela Microsoft.}
}

\newglossaryentry{Firm}
{
	name={Firmware},
	description={É o conjunto de instruções operacionais programadas directamente no \textit{hardware} de um equipamento electrónico.}
}

\newglossaryentry{browser}
{
	name={Browser},
	description={É um programa de computador que possibilita aos utilizadores uma interacção com documentos virtuais da Internet, também conhecidos como páginas Web.}
}

