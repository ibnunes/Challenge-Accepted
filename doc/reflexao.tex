\chapter{Reflexão Crítica e Problemas Encontrados}
\label{chap:reflexao}

\section{Introdução}
\label{chap4:sec:intro}

Não obstante o bom planeamento feito \textit{a priori} na fase de engenharia de \emph{software} (Capítulo \ref{ch::engsoft}), o projeto \appname~enfrentou contratempos, não tendo sido possível chegar a todas as ambições inicialmente imaginadas. Reflita-se, portanto, sobre o desenvolvimento deste projeto.

Neste Capítulo são explorados os seguintes tópicos:
\begin{itemize}
    \item Objetivos propostos vs. alcançados (Secção \ref{chap4:sec:opvsoa}): compara os objetivos
    inicialmente propostos com aqueles que foram concluídos no projeto final;
    \item Divisão de trabalho pelos elementos do grupo (Secção \ref{chap4:sec:divisao}): lista as tarefas
    realizadas por cada elemento da equipa;
    \item Problemas encontrados (Secção \ref{chap4:sec:problemas}): na sequência da Secção \ref{chap4:sec:opvsoa}, explora
    os problemas encontrados durante a implementação da aplicação;
    \item Reflexão crítica (Secção \ref{chap4:sec:reflexao}): é feita uma \ac{SWOT} em retrospetiva pela equipa acerca do projeto.
\end{itemize}


\section{Objetivos Propostos vs. Alcançados}
\label{chap4:sec:opvsoa}

A Tabela \ref{tab::objetivos} expõe os objetivos propostos inicialmente para o projeto e identifica quais foram alcançados na sua plenitude, quais foram alcançados parcialmente, e quais não tiveram sucesso.

\begin{table}[!htbp]
    \centering
    \begin{tabular}{p{.65\textwidth} >{\centering\let\newline\\\arraybackslash\hspace{0pt}}m{.25\textwidth}}
        \toprule
        {\bfseries Objetivo proposto} & {\bfseries Alcançado?} \\
        \midrule
        Registo de utilizadores com representação segura da palavra-passe na base de dados & $\bullet$ \\
        Submissão de desafios do tipo cifra de mensagens, codificando-a em \textit{BASE64} & $\bullet$ \\	
        Cálculo de código de autenticação de mensagens que permite verificar se uma mensagem foi bem decifradas & $\bullet$ \\
        Submissão de desafios do tipo valor de \textit{hash}, codificando-a em \textit{BASE64} & $\bullet$ \\
        Resposta a desafios e verificação do sucesso da tentativa  & $\bullet$ \\
        Desafios com cifras AES-128-ECB, AES-128-CBC e AES-128-CTR & $\bullet$ \\
        Desafios com funções de \textit{hash} MD5, SHA256 e SHA512          & $\bullet$ \\
        Limite de uma tentativa a cada 15 segundos para os desafios &  $\bullet$\\
        Verificar se uma mensagem foi bem decifrada através de assinaturas digitais RSA & -- \\
        Suporte ao algoritmo El Gamal                                           & -- \\
        Outros tipos de desafios criptográficos                                   & $\bullet$ \\
        \bottomrule
    \end{tabular}
    \caption[Objetivos propostos vs. alcançados]{
        Objetivos propostos e respetiva indicação de sucesso.\\
        \textit{Legenda.} $\bullet$ Alcançado em pleno; $\circ$ Alcançado parcialmente. -- Não alcançado.
    }
    \label{tab::objetivos}
\end{table}



\section{Divisão do Trabalho pelos Elementos do Grupo}
\label{chap4:sec:divisao}

\begin{table}[!htbp]
    \centering
    \begin{tabular}{l c c c c c}
        \toprule
        \textbf{Tarefa}                                      & \textbf{DL} & \textbf{JA} & \textbf{DS} & \textbf{BC} & \textbf{IN} \\
        \midrule
        Engenharia de \textit{Software}                      &             & $\circ$     &             & $\bullet$    &            \\
        Instalação da infraestrutura de suporte              & $\bullet$   &             &             &              & $\bullet$  \\
        Desenvolvimento da base de dados                     & $\bullet$   &             &             & $\circ$      &            \\
        \textit{Webservice} com autenticação da \textit{app} &             &             &             &              & $\bullet$  \\
        Implementação dos algoritmos de cifra                & $\bullet$   & $\bullet$   &             & $\bullet$    &            \\
        \textit{Refactoring} do código final                 &             &             & $\circ$     &              & $\bullet$  \\
        Validação de dados de utilizador                     &             &             & $\bullet$   &              &            \\
        Documentação do código                               &             &             & $\bullet$   &              &            \\
        Testes e tentativas de ataque                        & $\bullet$   &             & $\circ$     &              &            \\
        Gestão do repositório \textit{git}                   &             &             &             &              & $\bullet$  \\
        Relatório                                            &             & $\bullet$   &             & $\bullet$    & $\bullet$  \\
        Apresentação                                         &             & $\bullet$   &             & $\bullet$    &            \\
        \bottomrule
    \end{tabular}
    \caption[Distribuição de tarefas]{
        Distribuição de tarefas pelos elementos do grupo.\\
        \textit{Legenda.}~%
        $\bullet$ principal responsável; $\circ$ auxiliou.
        DL: Diogo Lavareda; JA: Joana Almeida; DS: Diogo Simões; BC: Beatriz Costa; IN: Igor Nunes.
    }
    \label{tab::divisao-trabalho}
\end{table}


\section{Problemas Encontrados}
\label{chap4:sec:problemas}

%\subsection{Rede \acs{eduroam} e certificado \acs{SSL}}
%\label{chap4:subsec:eduroam}

Nos primeiros testes à aplicação na rede \ac{eduroam}, onde o presente projeto será defendido, a equipa deparou-se com o problema desta rede bloquear as comunicações para o porto $3300$, utilizado na \acl{BD}.

Também devido ao facto deste servidor não ter um \ac{FQDN} para aplicar certificado \ac{SSL}, optou-se por separar o \emph{webservice} da \acl{BD} de forma a minimizar as alterações à estrutura do servidor e para permitir que a aplicação possa ser executada no ambiente de rede da \ac{UBI}. Como vantagem adicional, a plena separação do cliente e da base de dados com um \textit{webservice} intermédio permite a comunicação por \ac{HTTPS} (porto $443$) e adiciona uma camada de segurança com autenticação da aplicação que lhe acede (Figura \ref{fig::diagrama-sistema}).


\section{Reflexão Crítica}
\label{chap4:sec:reflexao}

Para reflexão da \groupname~face ao trabalho enveredado no desenvolvimento do serviço \appname, propõe-se efetivá-la com uma análise \ac{SWOT}.


\subsection{Pontos Fortes}
\label{chap4:subsec:pontosfortes}

\begin{enumerate}[nosep]
    \item A aplicação do cliente é \textit{cross-platform};
    \item São fornecidas seis cifras e três algoritmos de \textit{hash};
    \item Implementação de um sistema distribuído (arquitetura cliente -- \textit{webservice} -- \acl{BD});
    \item Apenas uma aplicação autorizada pode aceder ao \textit{webservice}.
\end{enumerate}


\subsection{Pontos Fracos}
\label{chap4:subsec:pontosfracos}

\begin{enumerate}[nosep]
    \item Falta de implementação das chaves de assinatura digital \ac{RSA};
    \item Falta da presença do algoritmo de \textit{El Gamal};
    \item A \acf{TUI} é demasiado minimalista.
\end{enumerate}


\subsection{Ameaças}
\label{chap4:subsec:ameacas}

\begin{enumerate}[nosep]
    \item Algumas dependências de baixo nível (\textit{e.g.} bibliotecas externas ao \textit{Python}) mudam entre sistemas operativos;
    \item As cifras \ac{AES} e os valores \textit{hash} são difíceis de resolver caso o utilizador não deixe uma dica útil (em último caso torna-se tecnicamente impossível de resolver);
    \item A \ac{BD} encontra-se exposta a acessos externos ao do \textit{webservice} na arquitetura atual.
\end{enumerate}



\subsection{Oportunidades}
\label{chap4:subsec:oportunidades}

\begin{enumerate}[nosep]
    \item Reforçar a segurança e verificação dos desafios através da assinatura digital \ac{RSA};
    \item Incluir novas cifras e outros algoritmos criptográficos;
    \item Implementação de uma \ac{GUI} e/ou de uma \textit{web app}.
\end{enumerate}



\section{Conclusões}
\label{chap4:sec:concs}

Esta fase de reflexão permitiu analisar o trabalho enveredado ao longo das semanas de planeamento, execução e teste. Com esta análise, a equipa pôde tirar conclusões não só sobre o seu desempenho, mas também acerca das tecnologias utilizadas, as quais serão expostas no Capítulo seguinte.
